\documentclass{cheatsheet}
\usepackage{bm}
\usepackage{textcomp, mathcomp}
\usepackage{empheq}
\usepackage{pbox}
\usepackage{booktabs}

\doctitle{Physik Zusammenfassung}
\author{Noa Sendlhofer \& Christian Leser \\ nsendlhofer \& cleser \\ \vspace*{-0.2em}}

% Noch hinzufügen: 
% Mechanische Leistung: P = \overrightarrow{F} \cdot \dot{\overrightarrow{x}}
% Mechanische Arbeit: \int\limits_{t_1}^{t_2} P(t) dt

% Hier ist eine Änderung

\begin{document}
\section{1. Elektrizitätslehre}
    \subsection{Ladung Q \hfill $[C]$}
    \begin{itemize}
        \item Elementarladung: $q_{Elektron} = e = - 1.602 \cdot 10^{-19}C$
    \end{itemize}

    Coulomb-Kraft: 

    \begin{minipage}{0.53\linewidth}
        \begin{footnotesize}
            \begin{center}
                \mathbox{
                    \vec{F_C}=\frac{1}{4\pi\varepsilon_0}\cdot \frac{q_1 \cdot q_2}
                    {r^2}\cdot \vec{e_r}            }
            \end{center}
        \end{footnotesize}
    \end{minipage}
    \begin{minipage}{0.46\linewidth}
        \begin{scriptsize}
            \begin{center}
                \begin{align*}
                    \varepsilon_0 &= 8,854\cdot10^{-12}
                    \\q_{1/2} &= \text{Punktladungen}
                    \\r &= \text{Abstand zw. Punktladungen}
                    \\\vec{e_r} &= \text{Einheitsvektor}
                \end{align*}
            \end{center}
        \end{scriptsize}
    \end{minipage}
    \vspace{1mm}



    \begin{itemize}
        \item Ladungen mit gleichem Vorzeichen stossen sich ab.
        \\$F_C<0 \rightarrow \text{abstossend}$,
        $F_C>0 \rightarrow \text{anziehend}$
        \item Ladungen leitender Körper stets an Oberfläche.\\
        $\rightarrow$ Inneres: Ladungs und Feldfrei
    \end{itemize}

    \subsubsection{Ladungsdichte}
        \begin{tabular}{c c c}
            Liniendichte $\lambda$ & Oberflächendichte $\sigma$ & Volumendichte $\rho$ \\
            $\lambda = \frac{Q}{l} \left[\frac{C}{m}\right]$ & $\sigma = \frac{Q}{A} \left[\frac{C}{m^2}\right]$ & $\rho = \frac{Q}{V} \left[\frac{C}{m^3}\right]$
        \end{tabular}


    \subsection*{1.2 Strom I [A]}
Strom A:

\vspace{-1mm}\begin{minipage}{0.4\linewidth}
    \begin{footnotesize}
        \begin{center}
            \mathbox{
                I = \frac{dQ}{dt} \left[\frac{C}{s}\right]
            }
            
        \end{center}
    \end{footnotesize}
\end{minipage}
$\longleftrightarrow$
\begin{minipage}{0.4\linewidth}
    \begin{footnotesize}
        \begin{center}
            \mathbox{
                Q = \int\limits_{\Delta t} I dt
            }
        \end{center}
    \end{footnotesize}
\end{minipage}
\vspace{1mm}

Stromdichte j:

\vspace{-1mm}\begin{minipage}{0.25\linewidth}
    \begin{footnotesize}
        \begin{center}
            \mathbox{
                j = \frac{I}{A} \left[\frac{A}{m^2}\right]
            }
            
        \end{center}
    \end{footnotesize}
\end{minipage}
$\longleftrightarrow$
\begin{minipage}{0.55\linewidth}
    \begin{footnotesize}
        \begin{center}
            \mathbox{
                I = \iint\limits_{A} j dA \overset{(\text{$I$ gleichm. auf $A$})}{=} j \cdot A
            }
        \end{center}
    \end{footnotesize}
\end{minipage}
\vspace{1mm}

\subsubsection{Stromdichte}
\begin{tabular}{c c c}
    & Flächendichte $j$ &\\
    & $j = \frac{I}{A} \left[\frac{C}{s \cdot m^2}\right]$ &
\end{tabular}


    \subsection{Elektrischer Widerstand R \hfill $[\Omega]$}
    Widerstand: \mathbox{R = \frac{U}{I}, I \sim U}

    \begin{tabular}{c c}
        Ohmsche Leiter & nicht-ohmsche Leiter \\
        $I = \frac{U}{R}$ & $R_{\text{diff}} = \frac{dU}{dI}$\\
        \includegraphics[width = 30mm]{src/images/plot_ohmscher_leiter.png} & \includegraphics[width = 30mm]{src/images/plot_nicht-ohmscher_leiter.png}
    \end{tabular}
    \vfill

    \subsubsection{Spezifischer Widerstand}
        nach Grösse:

        \vspace{-1mm}
        \begin{minipage}{0.49\linewidth}
            \begin{footnotesize}
                \begin{center}
                    \mathbox{
                        R = \rho \frac{l}{A}
                    }
                \end{center}
            \end{footnotesize}
        \end{minipage}
        \begin{minipage}{0.5\linewidth}
            \begin{scriptsize}
                \begin{center}
                    \begin{align*}
                        l &= \text{Leiterlänge}
                        \\A &= \text{Leiterquerschnitt} 
                        \\\rho &= \text{spezifischer Widerstand}
                        \\  K &= \frac{1}{\rho} = \text{Spezifische Leitfähigkeit}
                    \end{align*}
                \end{center}
            \end{scriptsize}
        \end{minipage}
        \vspace{1mm}

        nach Temperatur:

        \vspace{-1mm}
        \begin{minipage}{0.49\linewidth}
            \begin{footnotesize}
                \begin{center}
                    \mathbox{
                        \rho(T) = \rho_0 (1 + \alpha(T - T_0))
                    }
                \end{center}
            \end{footnotesize}
        \end{minipage}
        \begin{minipage}{0.5\linewidth}
            \begin{scriptsize}
                \begin{center}
                    \begin{align*}
                        \rho_0 &= \text{spezifischer Widerstand bei } T_0
                        \\T_0 &= \text{Bezugstemperatur}
                        \\\alpha &= \frac{1}{K} =\text{Temperaturkoeffizient}
                    \end{align*}
                \end{center}
            \end{scriptsize}
        \end{minipage}
        \vspace{1mm}
    \subsection{Elektrische Kapazität C \hfill $[F]$}
    \begin{minipage}{0.49\linewidth}
        \begin{center}
            Im Vakuum:
            \mathbox{C_0 = \frac{Q}{U} = \varepsilon_0 \frac{A}{l}}
        \end{center}
    \end{minipage}
    \begin{minipage}{0.49\linewidth}
        \begin{center}
            Materie (Dielektrikum mit $\varepsilon_m$):
            \begin{empheq}[box=\fbox]{align*}
                C_m &= \varepsilon_m C_0\\
                U_m &= \frac{U_0}{\varepsilon_m}\\
                \varepsilon_m &= \frac{\overrightarrow{E}_0}{\overrightarrow{E}_m}
            \end{empheq}
        \end{center}
    \end{minipage}

    \begin{minipage}{0.49\linewidth}
        \begin{center}
            Ladestrom Kondensator:
            \begin{empheq}[box=\fbox]{align*}
                I(t) &= I_0 \cdot e^{\left(-\frac{t}{R \cdot C}\right)}\\
                I(0) &= I_0 = \frac{U}{R_{\text{tot}}}\\
                I(\infty) &= 0
            \end{empheq}
        \end{center}
    \end{minipage}
    \begin{minipage}{0.49\linewidth}
        \begin{center}
            \begin{flushleft}
                \begin{scriptsize}
                    Einschieben von Dielektrika in einen Plattenkondensator: \\
                    Die Energieverteilung verändert sich bis ein Gleichgewichtszustand erreicht ist:
                \end{scriptsize}  
            \end{flushleft}
            \begin{empheq}[box=\fbox]{align*}
                0 &= dW_{\text{tot}}\\
                 &\scriptstyle= dW_{\text{Feld}} - dW_{\text{Batt}} + dW_{\text{Diel}}
            \end{empheq}
        \end{center}
    \end{minipage}
    \subsection*{1.5 Kirchhoffsche Regeln}
    \subsubsection*{Knotenregel}
    \vspace{-1mm}
    \begin{minipage}{0.49\linewidth}
        \begin{footnotesize}
            \begin{center}
                \vspace{2mm}
                \includegraphics[width = 30mm]{src/images/knotenregel.png}
            \end{center}
        \end{footnotesize}
    \end{minipage}
    \begin{minipage}{0.5\linewidth}
        \begin{scriptsize}
            \begin{center}
                \mathbox{
                    \sum\limits_k I_k = 0
                }
                \mathbox{
                    \sum\limits_k Q_k = 0
                }
            \end{center}
        \end{scriptsize}
    \end{minipage}
    \vspace{1mm}
    \hfill \colorbox{Goldenrod}{$\sum I_\text{zufliessend}$} $=$ \colorbox{Apricot}{$\sum I_\text{abfliessend}$} 


    \subsubsection*{Maschenregel}
    \vspace{-1mm}
    \begin{minipage}{0.49\linewidth}
        \begin{footnotesize}
            \begin{center}
                \vspace{2mm}
                \includegraphics[width = 30mm]{src/images/maschenregel.png}
            \end{center}
        \end{footnotesize}
    \end{minipage}
    \begin{minipage}{0.5\linewidth}
        \begin{scriptsize}
            \begin{center}
                \mathbox{
                    \sum\limits_{i} U_{i} = \sum\limits_k R_k I_k 
                }
                \mathbox{
                    \sum\limits_{i} U_{i} = \sum\limits_k \frac{Q_k}{C_k} 
                }
                $i =$ \# Spannungsquellen\\
                $k =$ \# Spannungsabfälle
            \end{center}
        \end{scriptsize}
    \end{minipage}
    \vspace{1mm}
    \hfill \colorbox{YellowGreen}{$\sum U_\text{Spannungsquelle}$} $=$ \colorbox{Yellow}{$\sum U_\text{Spannungsabfälle}$} 

    (1) Zeichne \colorbox{YellowGreen}{$\overrightarrow{U_{sq}}$}an der Spannungsquelle ein (minus nach plus)
    \\(2) Wähle Stromrichtung \colorbox{Cyan}{$\overrightarrow{I}$} (gegen $U_{sq}$)
    \\(3) Trage \colorbox{Yellow}{$\overrightarrow{U_R}$} an Widerständen (ein gleich wie Stromrichtung)
    \subsection*{1.6 Schaltkreis}
\vspace{-1mm}
\begin{minipage}{0.49\linewidth}
    \begin{footnotesize}
        \begin{center}
            \vspace{2mm}
            \includegraphics[width = 30mm]{src/images/schaltkreis.png}
        \end{center}
    \end{footnotesize}
\end{minipage}
\begin{minipage}{0.5\linewidth}
    \begin{scriptsize}
        \begin{center}
            \begin{align*}
                \vec{I} = &\text{ Stromrichtung}
                \\R = &\text{ Widerstand} 
                \\\vec{U} = &\text{ Richtung des Spannungsabfall}
                \\&\text{ Spannungsquelle: von mius nach plus}
                \\&\text{ Widerstand: in Stromrichtung}
            \end{align*}
        \end{center}
    \end{scriptsize}
\end{minipage}

\subsubsection*{Serieschaltung}
\vspace{-1mm}
\begin{minipage}{0.39\linewidth}
    \begin{footnotesize}
        \begin{center}
            \vspace{2mm}
            \includegraphics[width = 20mm]{src/images/serieschaltung.png}
        \end{center}
    \end{footnotesize}
\end{minipage}
\begin{minipage}{0.6\linewidth}
    \begin{scriptsize}
        \begin{center}
            \begin{align*}
                \text{Widerstände:} \; R_{res} &= \sum\limits_i R_i\\
                \text{Kondensatoren:} \; \frac{1}{C_{res}} &= \sum\limits_i \frac{1}{C_i}
            \end{align*}
        \end{center}
    \end{scriptsize}
\end{minipage}
\vspace{1mm}

\subsubsection*{Parallelschaltung}
\vspace{-1mm}
\begin{minipage}{0.39\linewidth}
    \begin{footnotesize}
        \begin{center}
            \vspace{2mm}
            \includegraphics[width = 20mm]{src/images/parallelschaltung.png}
        \end{center}
    \end{footnotesize}
\end{minipage}
\begin{minipage}{0.6\linewidth}
    \begin{scriptsize}
        \begin{center}
            \begin{align*}
                \text{Widerstände:} \; \frac{1}{R_{res}} &= \sum \frac{1}{R_i}\\
                \text{zwei Widerstände:} \; R_{res} &= \frac{R_1 \cdot R_2}{R_1 + R_2}\\
                \text{Kondensatoren:} \; C_{res} &= \sum\limits_i C_i
            \end{align*}
        \end{center}
    \end{scriptsize}
\end{minipage}

\section{2. Elektrostatik}
    \subsection{Elektrisches Feld}
    \begin{tabular}{l|l l}
        Homogenes Feld & Inhomogenes Feld & 
        \\(Plattenkondensator) & (Punktladung) & 
        \\
        \\\includegraphics[width = 10mm]{src/images/kondensator.png} & \includegraphics[width = 21mm]{src/images/zwei_punktladung.png} & \includegraphics[width = 15mm]{src/images/punktladung.png}
    \end{tabular}
    \vspace{2mm}

    Quick facts:
    \begin{itemize}
        \item Jede Ladung erzeugt ein $E$ - Feld \vspace{-1mm}
        \item Feldlinien von + nach - \vspace{-1mm}
        \item Feldlinien immer $\bot$ auf leitfähigen Körpern \vspace{-1mm}
        \item Feldlinien schneiden sich nie \vspace{-1mm}
        \item Innerhalb von Leitern gibt es kein Feld
    \end{itemize}


    \subsubsection*{Elektrische Feldstärke E $\left[\frac{V}{m}\right]$}
    \mathbox{E = -grad(\Phi) = - \frac{d \Phi(P)}{ds}}

    Inhomogen (Punktladung):
    \vspace{-1mm}
    \begin{minipage}{0.53\linewidth}
        \begin{footnotesize}
            \begin{center}
                \mathbox{
                    \overrightarrow{E} = \frac{\overrightarrow{F}}{q_2} = \frac{1}{4\pi\varepsilon_0} \cdot \frac{q_1}{| \overrightarrow{r}|^2} \cdot\overrightarrow{e}
                }
            \end{center}
        \end{footnotesize}
    \end{minipage}
    \vspace{1mm}
    \begin{minipage}{0.46\linewidth}
        \begin{scriptsize}
            \begin{center}
                \begin{align*}
                    \varepsilon_0 = &\enspace 8,854\cdot10^{-12}
                    \\\overrightarrow{r} = &\text{\enspace Abstand zw. Punktladungen}
                    \\&\text{\enspace(- nach +)}
                    \\\overrightarrow{e_r} = &\text{\enspace Einheitsvektor Richtung Feldlinien}
                \end{align*}
            \end{center}
        \end{scriptsize}
    \end{minipage}
    \vspace{2mm}
    $\longrightarrow$ Feld mehrer Punktladungen Summierbar (Vektorsumme)\\
    Homogene (Plattenkondensator):

    \begin{minipage}{0.53\linewidth}
        \begin{footnotesize}
            \begin{center}
                \mathbox{
                    \overrightarrow{E} = \frac{U}{l} \overrightarrow{e}
                }
            \end{center}
        \end{footnotesize}
    \end{minipage}
    \begin{minipage}{0.46\linewidth}
        \begin{scriptsize}
            \begin{center}
                \begin{align*}
                    l &= \text{Abstand der Platten}
                \end{align*}
            \end{center}
        \end{scriptsize}
    \end{minipage}
    \vspace{1mm}

    \textbf{Beispiel Spannungswaage}
    \subsubsection*{Elektrische Flussdichte /Verschiebungsdichte D$\left[\frac{V}{m}\right]$}
        Dichte der Ladung:
        \mathbox{\overrightarrow{D} = \frac{Q}{A} \overrightarrow{e}}
        Im Vakuum: 
        \mathbox{\overrightarrow{D} = \varepsilon_0 \cdot \overrightarrow{E}}
    \subsubsection*{Elektrisches Potential $\Phi$}
    \begin{tabular}{c c}
        Allgemein: & Punktladung:\\
        $\Phi(P) = -\int \overrightarrow{E} \overrightarrow{ds}$ & $\Phi(r) = \frac{1}{4 \pi \varepsilon_0} \frac{q_1}{r}$
    \end{tabular}
    \subsubsection{Elektrische Spannung U \hfill $[V]$}
        Inhomogen:
        \mathbox{U = \Phi(P) - \Phi(P_0) = -\int\limits_{P_0}^{P} \vec{E}\vec{ds}}
            
        Homogen:

        \vspace{-1mm}
            \begin{minipage}{0.41\linewidth}
                \begin{footnotesize}
                    \begin{center}
                        \vspace{2mm}
                        \includegraphics[width = 15 mm]{src/images/hom_potentialfeld.png}
                    \end{center}
                \end{footnotesize}
            \end{minipage}
            \begin{minipage}{0.58\linewidth}
                \begin{scriptsize}
                    \begin{center}
                        \mathbox{
                            U = E\cdot d
                        }
                        Auf \colorbox{Cyan}{Linie} stets selbes Potential/Spannung
                    \end{center}
                \end{scriptsize}
            \end{minipage}
            \vspace{1mm}
    \subsubsection{Elektrisches Dipolmoment}
    Für zwei gleich stark, mit unterschiedlichem Vorzeichen geladene Ladungen:\\
    \begin{minipage}{0.59\linewidth}
        \begin{empheq}[box = \fbox]{align*}
            \vec{p} &= Q \vec{l}\\
            \vec{E_{dip}} &= \frac{1}{4 \pi \varepsilon_0} \frac{3 (\vec{p} \hat{r}) \hat{r} - \vec{p}}{r^3}
        \end{empheq}
    \end{minipage}
    \begin{minipage}{0.39\linewidth}
        \begin{scriptsize}
            Variablen erklären
        \end{scriptsize}
    \end{minipage}

    %\input{src/2_elektrostatik/7_elektrischer_fluss.tex}
    \subsection{Kraft, Arbeit, Leistung}
    \subsubsection{Kraft im $\vec{E}$ Feld \hfill $[N]$}
        \mathbox{\vec{F_E} = Q \vec{E_0}}
    
    \subsubsection{pot. Energie von Ladungen \hfill $[J]$}
        \mathbox{E_{pot} = Q \Phi}
    
    %\vfill \null \columnbreak

    \subsubsection{Arbeit im $\vec{E}$ Feld \hfill $[J]$}
        Die gespeicherte Energie ist jeweils die verrichtete Arbeit: $\Delta E = W$
        Arbeit $=$ Kraft $\cdot$ Weg
        \begin{empheq}[box = \fbox]{align*}
            W &= \int \vec{F}d\vec{s} = \int \vec{E} \cdot Q d\vec{s} = Q \Delta \Phi\\
            &= Q U = C U^2 = U \cdot I \cdot t\\
            dW &= UI dt
        \end{empheq}
        \begin{center} \underline{\textbf{Gespeicherte Energie im Kondensator:}} \end{center}
        \begin{minipage}{0.49\linewidth}
            \begin{itemize}
                \item Energie entspricht Fläche unter Q-U-Diagramm.
                \item Ladevorgang eines Kondensators verläuft linear.
                \item Somit: Energie entspricht Dreiecksfläche mit Seitenlänge Ladung und Spannung nach Ladevorgang.
            \end{itemize}
        \end{minipage}
        \begin{minipage}{0.49\linewidth}
            \includegraphics[width = 1\linewidth]{src/images/ladevorgang_kondensator.png}
        \end{minipage}

        \begin{minipage}{0.69\linewidth}
            \begin{empheq}[box = \fbox]{align*}
                W &= \int U dq = \frac{Q U}{2} =  \frac{Q^2}{2C} = \frac{CU^2}{2}\\
                &= \frac{1}{2} \varepsilon_0 E_0^2 V\\
                \rho_{el} &= \frac{W}{V} = \frac{1}{2} \varepsilon_0 E_0^2
            \end{empheq}
        \end{minipage}
        \begin{minipage}{0.29\linewidth}
            \begin{scriptsize}
                \begin{empheq}{align*}
                    Q = &\text{Ladung}\\
                    U = &\text{Spannung}\\
                    C = &\text{Kapazität}\\
                    E_0 = &\text{Elektrische Feldstärke}\\
                    V = &\text{Volumen zwischen}\\
                    &\text{Kondensatorflächen}\\
                    \rho_{el} = &\text{Energiedichte}
                \end{empheq}
            \end{scriptsize}
        \end{minipage}
    \vfill \null \columnbreak

    \subsubsection{Leistung im $\vec{E}$ Feld \hfill $\left[\frac{J}{s}\right]$}
        \mathbox{P = \frac{W}{t} = F \cdot v = U \cdot I}
    
    \subsection*{Beispiele}
    \textbf{Beispiel Spannungswaage}
    % Fluss hab ich (Christian) bereits gemacht, kann es aber erst am Montag committen

\section{3. Magnetisches Feld}
    \subsection*{Magnetische Feldstärke H}
    \begin{minipage}{0.59\linewidth}
        \begin{itemize}
            \item Stromdurchflossene Leiter bauen magnetisches Feld auf
            \item Rechte-Hand-Regel
        \end{itemize}
        \centering \underline{\textbf{Spule}}\\
        \begin{minipage}{0.44\linewidth}
            \mathbox{\left| \overrightarrow{H} \right| = I \frac{n}{l}}
        \end{minipage}
%
        \begin{minipage}{0.54\linewidth}
            \begin{scriptsize}
                \begin{empheq}{align*}
                    H &= \text{Magnetische Feldstärke}\\
                    I &= \text{Stromstärke}\\
                    n &= \text{Windungen der Spule}\\
                    l &= \text{Länge der Spule}\\
                \end{empheq}
            \end{scriptsize}
        \end{minipage}
    \end{minipage}
%
    \begin{minipage}{0.39\linewidth}
        \includegraphics[width = \linewidth]{src/images/rechte_hand_magnetismus.png}
    \end{minipage}
    
    \subsection*{elektromagnetische Induktion}
    Strom in Spule wird durch Änderung des magnetischen Feldes erzeugt.
    Spannungsstoss
    \mathbox{S_U = \int U_i dt}
    $ S_u ~ \Delta H, S_u ~ n_i, S_u ~ A_i$
    
    \mathbox{\int U_i = \mu_0 n_i \Delta H A_i}
    $\mu_0$ magnetische Permabilität des Vakuums (siehe Skript)

    magnetischer Fluss $\Phi$, magnetische Flussdichte $B$:
    \mathbox{\Phi = \mu_0 A H \rightarrow B = \frac{\Phi}{A} = \mu_0 H}
    \mathbox{\overrightarrow{B} = \mu_0 \overrightarrow{H}}
    \mathbox{\Phi = \int \overrightarrow{B} \overrightarrow{dA}}

    Induktionsgesetz:
    \mathbox{U_i^{tot} = n_i \frac{d \Phi}{dt}}
    \mathbox{U_i = -\oint \overrightarrow{E} \overrightarrow{ds} = \frac{d}{dt} \int \overrightarrow{B} \overrightarrow{dA}}
    \subsection{3.3 Lenzsche Regel}
    Das system reagiert der Änderung des magnetischen Feldes entgegen / wehrt sich gegen die Änderung des magnetischen Feldes, funktioniert nur wenn eine induzierte Strommenge fliessen kann (bsp metallischer Ring)\\
    Richtung des induzierten Feldes ist entgegengesetzt der Änderung des äusseren Feldes.
    Induktionsgesetz:
    \mathbox{U_i^{tot} = n_i \frac{d \Phi}{dt}}
    \mathbox{U_i = \oint \overrightarrow{E} \overrightarrow{ds} = -\frac{d}{dt} \int \overrightarrow{B} \overrightarrow{dA}}
    \subsection*{3.4 Durchflutungsgesetz}
    \begin{minipage}{0.64\linewidth}
        \begin{empheq}[box = \fbox]{align*}
            \begin{array}{r@{\ }l} %{r@{\ }c@{\ }l}
                \oint\limits \vec{H} ds &= n \cdot I_{tot}\\
                &= \int \vec{j} + \frac{\vec{dD}}{dt} \vec{dA}\\
                rot(\vec{H}) &= \vec{j} + \dot{\vec{D}}
            \end{array}
        \end{empheq}  
    \end{minipage}
    \begin{minipage}{0.34\linewidth}
        \begin{scriptsize}
            \begin{empheq}{align*}
                \vec{H} &= \text{mag. Feldstärke}\\
                I &= j \cdot A = \text{el. Strom}\\
                j &= \text{Stromflächendichte}\\
                D &= \text{el. Flussdichte}
            \end{empheq}
        \end{scriptsize}
    \end{minipage}

    \subsubsection{Feldstärke eines geradlinigen Leiters}
        \begin{minipage}{0.64\linewidth}
            \begin{empheq}[box = \fbox]{align*}
                \text{In Leiter: } H(r) &= \frac{I}{2 \pi} \frac{r}{R^2}\\
                \text{Um Leiter: } H(r) &= \frac{I}{2 \pi r}
            \end{empheq}  
        \end{minipage}
        \begin{minipage}{0.34\linewidth}
            \begin{scriptsize}
                \begin{empheq}{align*}
                    H &= \text{mag. Feldstärke}\\
                    I &= \text{el. Strom}\\
                    R &= \text{Radius des el. Leiters}\\
                \end{empheq}
            \end{scriptsize}
        \end{minipage}
    \subsection*{3.5 Lorenzkraft}
    $l$ Länge des Stromdurchflossenen Leiters im Magnetfeld, $\overrightarrow{B}$ Magnetfeld
    \mathbox{\overrightarrow{F_L} = I (\overrightarrow{l} \times \overrightarrow{B})}
    
    mit $V = A \cdot l$ und $j = \frac{I}{A}$:
    \mathbox{\frac{\Delta F}{\Delta A} = \overrightarrow{j} \times \overrightarrow{B} \rightarrow \overrightarrow{F} = \int \overrightarrow{j} \times \overrightarrow{B} dV}

    mit $I = \rho A v$ und somit $\overrightarrow{j} = \rho \overrightarrow{v}$ (v Geschwindigkeit der Ladungen):
    \mathbox{\overrightarrow{F_L} = \int \rho (\overrightarrow{v} \times \overrightarrow{B}) dV = q (\overrightarrow{v} \times \overrightarrow{B})}
    ! Vorzeichen q!
    \subsection*{3.6 Biot-Savart Gesetz}
    Einfluss der magnetischen Wirkung eines stromdurchflossenen, beliebig geformten Mediums auf einen Punkt
    \mathbox{\overrightarrow{B} = \frac{\mu_0}{4 \pi} \int \frac{I \overrightarrow{dl} \times ^r}{r^2}}
    \subsection*{Selbstinduktion}
    \mathbox{U_i = -L \frac{dI}{dt}}
    \subsection*{3.8 Gegeninduktivität}
    Aus Induktionsgesetz für beide Spulen
    \mathbox{U_{i2} = L_{12} \frac{I_1}{dt}}
    Man kann die Rolle der Spulen tauschen (Feldspule und Induktionsspule) und erhält dieselbe Wirkung:
    \mathbox{L_{12} = L_{21}}
    Für übereinanderliegende Spulen gleicher Länge und gleichen Querschnitts:
    \mathbox{L_{12} = n_2 \frac{\Phi_1}{I_1} = \mu_0 n_1 n_2 \frac{A}{l}}
    \subsection*{3.9 Kraft und Arbeit im magnetischen Feld}
    Energie im magnetischen Feld:
    \mathbox{W = \int_0^{\infty} L \frac{dI}{dt} I dt = \frac{1}{2} L I_0^2 = \frac{1}{2} \mu_0 n^2 \frac{A}{l} I_0^2 = \frac{1}{2} \mu_0 V H^2}
    \subsection*{3.10 Magnetismus der Materie}
    Wird eine Materie mit magnetischen Eigenschaften in eine Induktionsspule eingefügt, so verstärkt sich die magnetische Wirkung um einen materialabhängigen Faktor $\mu$
    \mathbox{B_m = \mu \mu_0 H_0, L_m = \mu L_0}

    Magnetische Suszeptibilität:
    \mathbox{X = \mu - 1}

    Magnetisierung:
    \mathbox{\overrightarrow{M} = X \overrightarrow{H}}
    \begin{itemize}
        \item paramagnetische Materialien: $X > 0$, Magnetisierung in gleiche Richtung wie Feld. 
        \item diamagnetische Materialien: $X < 0$, Magnetisierung in entgegengesetzte Richtung wie Feld.
    \end{itemize}

    Elektronen bewegen sich auf einer Kreisbahn im Atom -> magnetisches Moment entsteht. Bei angelegtem magnetischem Feld werden die magnetischen Momente aller Atome parallel ausgerichtet
    \textbf{Schema mit magnetischem Moment einfügen}

    Hysterese: Wenn nach der Magnetisierung eines ferromagnetischen Materials das magnetische Feld wieder ausgeschalten wird, setzt ein "Memory-Effekt ein.
    Eine verbleibende magnetische Wirkung im Material bezeichnet man als \textbf{Remanenz}.
    Das Feld, welches benötigt wird, um die Remanenz auszulöschen, bezeichnet man als \textbf{Hoerzitivkraft}
    %\begin{figure}
        \includegraphics[height = 20mm]{src/images/permanentmagnet.png}
    %\end{figure}

    Meissner Effekt: Keine magnetischen Feldlinien treten in einen Supraleiter ein, perfektes diamagnetisches Verhalten.
    Anwendung: Magnet kann auf abgekühltem supraleiter-Material schweben, bsp. Magnetschwebebahn
    \subsection*{3.11 Wechselspannung}
    Dynamo (Wechselspannung durch sich drehende Schleife in homogenem Magnetfeld):
    \mathbox{U_i = \omega B_0 A_0 sin(\omega t) = U_0 sin(\omega t)}
    \mathbox{I(t) = I_0 (cos(\omega t) + i sin(\omega t)) = I_0 e^{i \omega t}}

    Widerstand: Phasenverschiebung $U(t) \sim I(t)$
    \mathbox{U_R(t) = R I(t) = Z_R I(t) e^{i \omega t} \text{ mit } Z_R = R}

    Spule: Phasenverschiebung $U(t) \sim I(t + \frac{\pi}{2})$
    \mathbox{U_L = L \frac{dI}{dt} = Z_L I(t) e^{i \omega t} \text{ mit } Z_L = i \omega L}

    Kondensator: Phasenverschiebung $U(t) \sim I(t - \frac{\pi}{2})$
    \mathbox{U_C = \frac{1}{C} \int I(t) dt = Z_C I(t) e^{i \omega t} \text{ mit } Z_C = \frac{1}{i \omega C}}

    Impedanz:
    \mathbox{Z = Re + i Im = |Z| e^{i \phi}}
    \mathbox{U(t) = Z I(t)}
    \mathbox{Z_{\text{tot}} = \sum_i Z_i, \frac{1}{Z_{\text{tot}}} = \sum_i \frac{}{Z_i}}

    Leistung in komplexer Schreibweise ($\rho = $ Phasenverschiebung):
    \mathbox{\frac{1}{2} I_0 U_0 cos(\rho)}

    \subsection*{3.12 Funktionsweise eines Transformators}
    \centering
    \includegraphics[height = 30mm]{src/images/transformator.png}

    \mathbox{\left| \frac{U_p}{U_s} \right| = \frac{n_p}{n_s}}

\section{4. Elektromagnetische Wellen}
    \subsection{Herzscher Dipol}
    Elektrisches Pendel: Kondensator und Spule sind Energiespeicher.
    Wenn das magnetische Feld abgebaut wird, so wird das elektrische aufgebaut und umgekehrt.
    Idealisiert (ohne Reibung) "pendelt" dieses System unendlich lange
    \centering
    \includegraphics[height = 30mm]{src/images/herzscher_dipol.png}

    Dipolmoment $\vec{p}$ (alternierende Richtung durch Wechselspannung):
    \mathbox{\vec{p} = q \vec{l} = \vec{p_0} cos(\omega t)}

    Fernfeld: Entfernt man sich weit vom Sender (Herzschen Dipol), so verschwindet der phasenunterschied zwischen elektrischem und magnetischem Feld.

    \mathbox{\vec{E} = \vec{E_0} cos(\omega t - k r)}
    \mathbox{\vec{H} = \vec{H_0} cos(\omega t - k r)}
    mit Kreisfrequenz $\omega = 2 \pi \nu \rightarrow T = \frac{1}{\nu}$, Wellenzahl $k = \frac{2 \pi}{\lambda}$
    [$\nu = $ Frequenz, $T = $ Periode, $\lambda = $ Wellenlänge]
    \vfill \null \columnbreak

    \subsection*{4.2 Wellengleichung}
    \mathbox{\frac{\partial^2 E}{\partial t^2} = \frac{1}{\varepsilon_0 \mu_0} \frac{\partial^2 E}{\partial z^2}}
    \mathbox{\frac{\partial^2 H}{\partial t^2} = \frac{1}{\varepsilon_0 \mu_0} \frac{\partial^2 H}{\partial z^2}} 
    \subsection{4.3 Poynting-Vektor}
    Energiestromdichte und Poynting-Vektor:
    \mathbox{\vec{S} = \vec{E} \times \vec{H} = \vec{j_E}}

    Energiedichte:
    \mathbox{|\vec{j_E}| \approx \rho_E = \frac{1}{2} \varepsilon_0 E^2 + \frac{1}{2} \mu_0 H^2}
    \subsection*{4.4 Wellen}
    Höhe $\Phi$:
    \mathbox{\Psi (z, t) = A cos(\omega t - k z)}
    Wellenzahl $k = \frac{2 \pi}{\lambda}$, $z$ bezeichnet die entfernung zum Ursprung der cosinusfunktion zu Zeitpunkt $t_0$

    Phasengeschwindigkeit: Höhe konstant und somit Argument des cos() konstant:
    \mathbox{\omega t - k z = const}
    \mathbox{v_{ph} = c = f \lambda = \frac{\omega}{k}}
    Frequenz $f$

    Gruppengeschwindigkeit:
    \mathbox{v_{gr} = \frac{d \omega}{dk}}


    Intensität einer Welle: Energie
    Potentielle Energie:
    \mathbox{\Delta E_p = \int \overrightarrow{F} \overrightarrow{dx} = \frac{1}{2} D x_0^2 \text{mit} D = \omega^2 m}
    Kinetische Energie:
    \mathbox{\Delta E_k = \frac{1}{2} \Delta m v^2 \text{mit} v = \omega s}
    $\rightarrow \Delta E_p = \Delta E_k$  

    Energiestromdichte:
    \mathbox{\overrightarrow{j_E} = \frac{1}{A} \frac{\Delta E_k}{\Delta t} = \rho_E \overrightarrow{v_ph}}

    Reflektion von Wellen:
    Phasensprung bei Welle im Seil mittels Superposition mit einer Welle von der anderen Seite
    \centering
    \includegraphics[height = 30mm]{src/images/welle_superposition.png}

    \subsection{4.5 Doppler-Effekt}
    \mathbox{\lambda_0 = \frac{c}{f_0}}
    
    Wenn Empfänger sich auf Sender zubewegt:
    \mathbox{f' = f_0 (1 + \frac{v}{c})}

    Wenn Empfänger sich von Sender wegbewegt:
    \mathbox{f' = f_0 (1 - \frac{v}{c})}

    Wenn Sender sich auf Empfänger zubewegt:
    \mathbox{f' = \frac{1}{1 - \frac{v}{c}}}

    Wenn Sender sich von Empfänger wegbewegt:
    \mathbox{f' = \frac{1}{1 + \frac{v}{c}}}

    Für $v << c$ gilt, dass es nicht drauf ankommt ob Sender oder Empfänger sich in Ruhe befindet.

    Für Licht im Vakuum:
    $\beta = \frac{v}{c}$

    Quelle und Empfänger entfernen sich (Redshift):
    \mathbox{f' = f_0 \sqrt{\frac{1 - \beta}{1 + \beta}}}

    Quelle und Empfänger nähern sich (Blueshift):
    \mathbox{f' = f_0 \sqrt{\frac{1 + \beta}{1 - \beta}}}
    % Additionstheoreme $\cos(\alpha) + \cos(\beta) = 2 cos(\frac{\alpha + \beta}{2}) cos(\frac{\alpha - \beta}{2})$

\section{5. Appendix}
    \subsection{5.1 LR-Kreise}
    LR-Kreise bestehen aus einer Gleichstromquelle, einem Schalter, einer Spule und einem Widerstand in Reihe geschalten.\\
    Maschenregel:
    \mathbox{U_0 = U_R(t) + U_L(t)}

    mit $U_R(t) = R I(t)$ und $U_L(t) = L \frac{dI(t)}{dt}$ und geteilt durch $L$
    DGL der Stromstärke und ihre Lösung:\\
    Einschalten:
    \mathbox{\dot{I}(t) + \frac{R}{L} I(t) = \frac{U_0}{L}, I(t) = \frac{U_0}{R} \left(1 - e^{-\frac{R}{L}(t - t_0)}\right)}
    Ausschalten ($U_0 = 0$):
    \mathbox{I(t) = \frac{U_0}{R} \left(e^{-\frac{R}{L}(t - t_0)}\right)}

\section{6. Variabeln und Konstanten}
    \subsection*{Basics}
    \begin{itemize}
        \item Elementarladung: $e = 1,6 \cdot 10^{-19} C$
        \item Ampere: Fluss von 1 Coulomb pro Sekunde durch Leiterquerschnitt, $A = \frac{C}{s}$
        \item Newton: Kraft, $N = \frac{kg \cdot m}{s^2}$
        \item Volt: $V = \frac{W}{A} = \frac{J}{C} = \frac{kg \cdot m^2}{A \cdot s^3}$, Spannung entsteht durch Potentialdifferenz: $U \sim \int \overrightarrow{E} \cdot \overrightarrow{ds}$
        \item Elektrische Feldkonstante / Dielektrizitätskonstante $\varepsilon_0 = 8.854 \cdot 10^{-12} \frac{As}{Vm}$
        \item Magnetische Feldkonstante bzw. magnetische Permeabilität im Vakuum: $\mu_0 = \frac{1}{\varepsilon_0 c^2} = 4 \pi \cdot 10^{-7} \frac{N}{A^2}$
        \item Zentripetalkraft: $F_Z = m \frac{v^2}{r} = m \omega^2 r$
        \item Schwingungsdauer (Zeit für durchlauf einer vollen Welle) $T$
        \item Wellenlänge $\lambda = c T = \frac{c}{f}$
        \item Kreisfrequenz: $\omega = 2 \pi f = \frac{2 \pi}{T}$
        \item Frequenz $f = \nu = \frac{1}{T}, [f] = Hz = s^{-1}$
        \item Lichtgeschwindigkeit $c \approx 3 \cdot 10^8 \frac{m}{s}$
    \end{itemize}
    
    \begin{tabular}{c c}
        \textbf{Symbol} & \textbf{Exponent}\\
%        Y & $10^{24}$ \\
%        Z & $10^{21}$ \\
%        E & $10^{18}$ \\
        P & $10^{15}$ \\
        T & $10^{12}$ \\
        G & $10^9$ \\
        M & $10^6$ \\
        k & $10^3$ \\
        h & $10^2$ \\
%        da & $10^1$ \\
        d & $10^{-1}$ \\
        c & $10^{-2}$ \\
        m & $10^{-3}$ \\
        $\mu$ & $10^{-6}$ \\
        n & $10^{-9}$ \\
        p & $10^{-12}$ \\
        f & $10^{-15}$ \\
%        a & $10^{-18}$ \\
%        z & $10^{-21}$ \\
%        y & $10^{-24}$ 
    \end{tabular}
\end{document}