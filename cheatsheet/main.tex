\documentclass{cheatsheet}
\usepackage{bm}
\usepackage{textcomp, mathcomp}
\usepackage{empheq}
\usepackage{pbox}
\usepackage{booktabs}

\doctitle{Physik Zusammenfassung}
\author{Noa Sendlhofer \& Christian Leser \\ nsendlhofer \& cleser \\ \vspace*{-0.2em}}

% Noch hinzufügen: 
% Mechanische Leistung: P = \overrightarrow{F} \cdot \dot{\overrightarrow{x}}
% Mechanische Arbeit: \int\limits_{t_1}^{t_2} P(t) dt

% Hier ist eine Änderung

\begin{document}
\section{1. Elektrizitätslehre}
    \subsection*{Basics}
    \begin{itemize}
        \item Elementarladung: $e = 1,6 \cdot 10^{-19} C$
        \item Ampere: Fluss von 1 Coulomb pro Sekunde durch Leiterquerschnitt, $A = \frac{C}{s}$
        \item Newton: Kraft, $N = \frac{kg \cdot m}{s^2}$
        \item Volt: $V = \frac{W}{A} = \frac{J}{C} = \frac{kg \cdot m^2}{A \cdot s^3}$, Spannung entsteht durch Potentialdifferenz: $U \sim \int \overrightarrow{E} \cdot \overrightarrow{ds}$
        \item Elektrische Feldkonstante / Dielektrizitätskonstante $\varepsilon_0 = 8.854 \cdot 10^{-12} \frac{As}{Vm}$
        \item Magnetische Feldkonstante bzw. magnetische Permeabilität im Vakuum: $\mu_0 = \frac{1}{\varepsilon_0 c^2} = 4 \pi \cdot 10^{-7} \frac{N}{A^2}$
        \item Zentripetalkraft: $F_Z = m \frac{v^2}{r} = m \omega^2 r$
        \item Schwingungsdauer (Zeit für durchlauf einer vollen Welle) $T$
        \item Wellenlänge $\lambda = c T = \frac{c}{f}$
        \item Kreisfrequenz: $\omega = 2 \pi f = \frac{2 \pi}{T}$
        \item Frequenz $f = \nu = \frac{1}{T}, [f] = Hz = s^{-1}$
        \item Lichtgeschwindigkeit $c \approx 3 \cdot 10^8 \frac{m}{s}$
    \end{itemize}
    
    \begin{tabular}{c c}
        \textbf{Symbol} & \textbf{Exponent}\\
%        Y & $10^{24}$ \\
%        Z & $10^{21}$ \\
%        E & $10^{18}$ \\
        P & $10^{15}$ \\
        T & $10^{12}$ \\
        G & $10^9$ \\
        M & $10^6$ \\
        k & $10^3$ \\
        h & $10^2$ \\
%        da & $10^1$ \\
        d & $10^{-1}$ \\
        c & $10^{-2}$ \\
        m & $10^{-3}$ \\
        $\mu$ & $10^{-6}$ \\
        n & $10^{-9}$ \\
        p & $10^{-12}$ \\
        f & $10^{-15}$ \\
%        a & $10^{-18}$ \\
%        z & $10^{-21}$ \\
%        y & $10^{-24}$ 
    \end{tabular}
    \subsection*{1.1 Definition Strom}

Ladungen mit gleichem Vorzeichen stossen sich ab.\\
zwei unendlich lange parallele Drähte im Abstand 1 m voneinander, die von einem Strom von 1 A gleichsinnig durchflossen werden, ziehen sich mit einer Kraft von $2x10^{-7} N$ pro Meter Leiterlänge an.\\
Elementarladung: $e = 1.602 \cdot 10^{-19}$

\mathbox{I = \frac{dQ}{dt}}
\mathbox{Q = \int\limits_{\Delta t} I dt}

\mathbox{R = \frac{U}{I}, I \sim U}

\begin{tabular}{c c}
    Ohmsche Leiter & nicht-ohmsche Leiter \\
    $I = \frac{U}{R}$ & $R_{\text{diff}} = \frac{dU}{dI}$\\
    plot 1 & plot 2
\end{tabular}


    \subsection*{1.2 Klassifizierung ohmscher Leiter}
nach Grösse:
\mathbox{R = \rho \frac{l}{A}}, l Leiterlänge, A Leiterquerschnitt, $\rho$ spezifischer Widerstand
Spezifische Leitfähigkeit: $ K = \frac{1}{\rho}$\\
nach Temperatur:
\mathbox{\rho(T)}
    \subsection*{1.3 Kirchhoffsche Regeln}
    \subsubsection*{Knotenregel}
    \mathbox{\sum\limits_k I_k = 0}

    \subsubsection*{Maschenregel}
    \mathbox{\sum\limits_i U_i = \sum\limits_k I_k R_k}

    \subsubsection*{Serieschaltung}
    \[
        U_0 = I \cdot R_{tot} = I \cdot \left(\sum\limits_i R_i\right) \Rightarrow \boxed{R_{tot} = \sum\limits_i R_i}
    \]

    \subsubsection*{Parallelschaltung}
    \[
        I_{tot} = \frac{U}{R_{tot}} = \sum\limits_i \frac{U}{R_i} \Rightarrow \boxed{\frac{1}{R_{tot}} = \sum\limits_i \frac{1}{R_i}}
    \]

\section{2. Elektrisches Feld}
    \subsection{2.1 Feldlinien}
    Elektrisches Feld immer tangential an Feldlinie
    \begin{tabular}{c c c}
        \includegraphics[height = 20mm]{src/images/punktladung.png} & \includegraphics[height = 20mm]{src/images/zwei_punktladung.png} & \includegraphics[height = 30mm]{src/images/kondensator.png}
    \end{tabular}

    \subsubsection*{Elektrische Feldstärke}
        \mathbox{\overrightarrow{E} = \frac{U}{l} \overrightarrow{e}} mit e Einheitsvektor in Richtung der Feldlinien

    \subsubsection*{Verschiebungsdichte}
        \mathbox{\overrightarrow{D} = \frac{Q}{A} \overrightarrow{e}}
        Im Vakuum: $\overrightarrow{D} = \epsilon_0 \cdot \overrightarrow{E}$\\
        Elektrische Feldkonstante $\epsilon_0 = 8.854 \cdot 10^{-12} \frac{As}{Vm}$
    \subsection*{Elektrisches Potential}
Spannung: Potentialdifferenz
\begin{align*}
    & U = \Psi(p1) - \Psi(p0) = -\int E ds\\
    & E = -grad(\Psi)\\
    & \text{Für Kugel: } \Psi(r) = \frac{1}{4 \pi \varepsilon_0} \frac{Q}{r}\\
    & \text{Potential einzelner Punkt in räumlicher Ladungsverteilung: } \Psi(P) = \frac{1}{4 \pi \varepsilon_0} \int \frac{\rho dV}{r}\\
    & \text{Dipolmoment: wie F * l: } \overrightarrow{p} = Q \overrightarrow{l}\\
    & \text{Dipolpotential} \Psi_{dip} = \frac{1}{4 \pi \varepsilon_0} \frac{\overrightarrow{p} \hat{r}}{r^2}\\
\end{align*}
    \input{src/2_elektrisches_feld/3_elektrischer_fluss.tex}
    \subsection*{2.3 Plattenkondensator}
    \mathbox{C_0 = \frac{Q}{U} = \varepsilon_0 \frac{A}{l}, [C] = F = \text{Farad}}
    Mit Dielektrikum gefüllter Plattenkondensator:
    \mathbox{C_m = \varepsilon_m C_0 \xrightarrow{Q = const} U_m = \frac{U_0}{\varepsilon_m}, E_m = \frac{E_0}{\varepsilon_mE_0}{\varepsilon_m}}
    Kugel:
    \mathbox{C = 4 \pi \varepsilon_0 \left( \frac{r_1 r_2}{r_2 - r_1} \right)}
    Kirchhoffsche Regeln in Kondensatoren:
    Knotenregel:
    \mathbox{\sum_k Q_k = 0}
    Maschenregel:
    \mathbox{\sum_i = U_i = \sum_k \frac{Q_k}{C_k}}
    Serieschaltung:
    \mathbox{\frac{1}{C} = \frac{1}{C_1} + \frac{1}{C_2}}
    Parallelschaltung:
    \mathbox{C = C_1 + C_2}

    Ladestrom des Kondensators: $I = I_0 e^{-\frac{t}{RC}}$, somit erreicht der Kondensator niemals seine maximale Kapazität
    \subsection*{Kraft und Arbeit im elekrtischen Feld}
    \mathbox{\overrightarrow{F} = Q \overrightarrow{E_0}}
    W = F * l Arbeit ist Kraft mal Weg
    \mathbox{W = \int \overrightarrow{F}d\overrightarrow{s} = -\int Q \overrightarrow{E_0}d\overrightarrow{s} = Q \Delta \Phi = QU = U \cdot I \cdot t}
    \mathbox{dW = UI dt}
    Momentanleistung P:
    \mathbox{P = \frac{W}{t} = \frac{F}{v} = \frac{dW}{dt} = U \cdot I}
    Energie des elektrischen Feldes: $\Delta E = W$
    \mathbox{W = \int U dq =  \frac{Q^2}{2C} = \frac{CU^2}{2}} mit U = Q/C (von Kondensatoren)
    Mit Kapazität eines Plattenkondensators (V = Volumen des Plattenkondensators)
    \mathbox{W = \frac{1}{2} \varepsilon_0 E_0^2 V}
    \mathbox{\rho_{el} = \frac{W}{V} = \frac{1}{2} \varepsilon_0 E_0^2}

\section{3. Magnetisches Feld}
    \subsection*{Magnetische Feldstärke H}
    \begin{minipage}{0.59\linewidth}
        \begin{itemize}
            \item Stromdurchflossene Leiter bauen magnetisches Feld auf
            \item Rechte-Hand-Regel
        \end{itemize}
        \centering \underline{\textbf{Spule}}\\
        \begin{minipage}{0.44\linewidth}
            \mathbox{\left| \overrightarrow{H} \right| = I \frac{n}{l}}
        \end{minipage}
%
        \begin{minipage}{0.54\linewidth}
            \begin{scriptsize}
                \begin{empheq}{align*}
                    H &= \text{Magnetische Feldstärke}\\
                    I &= \text{Stromstärke}\\
                    n &= \text{Windungen der Spule}\\
                    l &= \text{Länge der Spule}\\
                \end{empheq}
            \end{scriptsize}
        \end{minipage}
    \end{minipage}
%
    \begin{minipage}{0.39\linewidth}
        \includegraphics[width = \linewidth]{src/images/rechte_hand_magnetismus.png}
    \end{minipage}
    
    \subsection*{elektromagnetische Induktion}
    Strom in Spule wird durch Änderung des magnetischen Feldes erzeugt.
    Spannungsstoss
    \mathbox{S_U = \int U_i dt}
    $ S_u ~ \Delta H, S_u ~ n_i, S_u ~ A_i$
    
    \mathbox{\int U_i = \mu_0 n_i \Delta H A_i}
    $\mu_0$ magnetische Permabilität des Vakuums (siehe Skript)

    magnetischer Fluss $\Phi$, magnetische Flussdichte $B$:
    \mathbox{\Phi = \mu_0 A H \rightarrow B = \frac{\Phi}{A} = \mu_0 H}
    \mathbox{\overrightarrow{B} = \mu_0 \overrightarrow{H}}
    \mathbox{\Phi = \int \overrightarrow{B} \overrightarrow{dA}}

    Induktionsgesetz:
    \mathbox{U_i^{tot} = n_i \frac{d \Phi}{dt}}
    \mathbox{U_i = -\oint \overrightarrow{E} \overrightarrow{ds} = \frac{d}{dt} \int \overrightarrow{B} \overrightarrow{dA}}
    \subsection{3.3 Lenzsche Regel}
    Das system reagiert der Änderung des magnetischen Feldes entgegen / wehrt sich gegen die Änderung des magnetischen Feldes, funktioniert nur wenn eine induzierte Strommenge fliessen kann (bsp metallischer Ring)\\
    Richtung des induzierten Feldes ist entgegengesetzt der Änderung des äusseren Feldes.
    Induktionsgesetz:
    \mathbox{U_i^{tot} = n_i \frac{d \Phi}{dt}}
    \mathbox{U_i = \oint \overrightarrow{E} \overrightarrow{ds} = -\frac{d}{dt} \int \overrightarrow{B} \overrightarrow{dA}}
    \subsection*{3.4 Durchflutungsgesetz}
    \begin{minipage}{0.64\linewidth}
        \begin{empheq}[box = \fbox]{align*}
            \begin{array}{r@{\ }l} %{r@{\ }c@{\ }l}
                \oint\limits \vec{H} ds &= n \cdot I_{tot}\\
                &= \int \vec{j} + \frac{\vec{dD}}{dt} \vec{dA}\\
                rot(\vec{H}) &= \vec{j} + \dot{\vec{D}}
            \end{array}
        \end{empheq}  
    \end{minipage}
    \begin{minipage}{0.34\linewidth}
        \begin{scriptsize}
            \begin{empheq}{align*}
                \vec{H} &= \text{mag. Feldstärke}\\
                I &= j \cdot A = \text{el. Strom}\\
                j &= \text{Stromflächendichte}\\
                D &= \text{el. Flussdichte}
            \end{empheq}
        \end{scriptsize}
    \end{minipage}

    \subsubsection{Feldstärke eines geradlinigen Leiters}
        \begin{minipage}{0.64\linewidth}
            \begin{empheq}[box = \fbox]{align*}
                \text{In Leiter: } H(r) &= \frac{I}{2 \pi} \frac{r}{R^2}\\
                \text{Um Leiter: } H(r) &= \frac{I}{2 \pi r}
            \end{empheq}  
        \end{minipage}
        \begin{minipage}{0.34\linewidth}
            \begin{scriptsize}
                \begin{empheq}{align*}
                    H &= \text{mag. Feldstärke}\\
                    I &= \text{el. Strom}\\
                    R &= \text{Radius des el. Leiters}\\
                \end{empheq}
            \end{scriptsize}
        \end{minipage}
    \subsection*{3.5 Lorenzkraft}
    $l$ Länge des Stromdurchflossenen Leiters im Magnetfeld, $\overrightarrow{B}$ Magnetfeld
    \mathbox{\overrightarrow{F_L} = I (\overrightarrow{l} \times \overrightarrow{B})}
    
    mit $V = A \cdot l$ und $j = \frac{I}{A}$:
    \mathbox{\frac{\Delta F}{\Delta A} = \overrightarrow{j} \times \overrightarrow{B} \rightarrow \overrightarrow{F} = \int \overrightarrow{j} \times \overrightarrow{B} dV}

    mit $I = \rho A v$ und somit $\overrightarrow{j} = \rho \overrightarrow{v}$ (v Geschwindigkeit der Ladungen):
    \mathbox{\overrightarrow{F_L} = \int \rho (\overrightarrow{v} \times \overrightarrow{B}) dV = q (\overrightarrow{v} \times \overrightarrow{B})}
    ! Vorzeichen q!
    \subsection*{3.6 Biot-Savart Gesetz}
    Einfluss der magnetischen Wirkung eines stromdurchflossenen, beliebig geformten Mediums auf einen Punkt
    \mathbox{\overrightarrow{B} = \frac{\mu_0}{4 \pi} \int \frac{I \overrightarrow{dl} \times ^r}{r^2}}
    \subsection*{Selbstinduktion}
    \mathbox{U_i = -L \frac{dI}{dt}}
    \subsection*{3.8 Gegeninduktivität}
    Aus Induktionsgesetz für beide Spulen
    \mathbox{U_{i2} = L_{12} \frac{I_1}{dt}}
    Man kann die Rolle der Spulen tauschen (Feldspule und Induktionsspule) und erhält dieselbe Wirkung:
    \mathbox{L_{12} = L_{21}}
    Für übereinanderliegende Spulen gleicher Länge und gleichen Querschnitts:
    \mathbox{L_{12} = n_2 \frac{\Phi_1}{I_1} = \mu_0 n_1 n_2 \frac{A}{l}}
    \subsection*{3.9 Kraft und Arbeit im magnetischen Feld}
    Energie im magnetischen Feld:
    \mathbox{W = \int_0^{\infty} L \frac{dI}{dt} I dt = \frac{1}{2} L I_0^2 = \frac{1}{2} \mu_0 n^2 \frac{A}{l} I_0^2 = \frac{1}{2} \mu_0 V H^2}
    \subsection*{3.10 Magnetismus der Materie}
    Wird eine Materie mit magnetischen Eigenschaften in eine Induktionsspule eingefügt, so verstärkt sich die magnetische Wirkung um einen materialabhängigen Faktor $\mu$
    \mathbox{B_m = \mu \mu_0 H_0, L_m = \mu L_0}

    Magnetische Suszeptibilität:
    \mathbox{X = \mu - 1}

    Magnetisierung:
    \mathbox{\overrightarrow{M} = X \overrightarrow{H}}
    \begin{itemize}
        \item paramagnetische Materialien: $X > 0$, Magnetisierung in gleiche Richtung wie Feld. 
        \item diamagnetische Materialien: $X < 0$, Magnetisierung in entgegengesetzte Richtung wie Feld.
    \end{itemize}

    Elektronen bewegen sich auf einer Kreisbahn im Atom -> magnetisches Moment entsteht. Bei angelegtem magnetischem Feld werden die magnetischen Momente aller Atome parallel ausgerichtet
    \textbf{Schema mit magnetischem Moment einfügen}

    Hysterese: Wenn nach der Magnetisierung eines ferromagnetischen Materials das magnetische Feld wieder ausgeschalten wird, setzt ein "Memory-Effekt ein.
    Eine verbleibende magnetische Wirkung im Material bezeichnet man als \textbf{Remanenz}.
    Das Feld, welches benötigt wird, um die Remanenz auszulöschen, bezeichnet man als \textbf{Hoerzitivkraft}
    %\begin{figure}
        \includegraphics[height = 20mm]{src/images/permanentmagnet.png}
    %\end{figure}

    Meissner Effekt: Keine magnetischen Feldlinien treten in einen Supraleiter ein, perfektes diamagnetisches Verhalten.
    Anwendung: Magnet kann auf abgekühltem supraleiter-Material schweben, bsp. Magnetschwebebahn
    \subsection*{3.11 Wechselspannung}
    Dynamo (Wechselspannung durch sich drehende Schleife in homogenem Magnetfeld):
    \mathbox{U_i = \omega B_0 A_0 sin(\omega t) = U_0 sin(\omega t)}
    \mathbox{I(t) = I_0 (cos(\omega t) + i sin(\omega t)) = I_0 e^{i \omega t}}

    Widerstand: Phasenverschiebung $U(t) \sim I(t)$
    \mathbox{U_R(t) = R I(t) = Z_R I(t) e^{i \omega t} \text{ mit } Z_R = R}

    Spule: Phasenverschiebung $U(t) \sim I(t + \frac{\pi}{2})$
    \mathbox{U_L = L \frac{dI}{dt} = Z_L I(t) e^{i \omega t} \text{ mit } Z_L = i \omega L}

    Kondensator: Phasenverschiebung $U(t) \sim I(t - \frac{\pi}{2})$
    \mathbox{U_C = \frac{1}{C} \int I(t) dt = Z_C I(t) e^{i \omega t} \text{ mit } Z_C = \frac{1}{i \omega C}}

    Impedanz:
    \mathbox{Z = Re + i Im = |Z| e^{i \phi}}
    \mathbox{U(t) = Z I(t)}
    \mathbox{Z_{\text{tot}} = \sum_i Z_i, \frac{1}{Z_{\text{tot}}} = \sum_i \frac{}{Z_i}}

    Leistung in komplexer Schreibweise ($\rho = $ Phasenverschiebung):
    \mathbox{\frac{1}{2} I_0 U_0 cos(\rho)}

    \subsection*{3.12 Funktionsweise eines Transformators}
    \centering
    \includegraphics[height = 30mm]{src/images/transformator.png}

    \mathbox{\left| \frac{U_p}{U_s} \right| = \frac{n_p}{n_s}}

\section{4. Elektromagnetische Wellen}
    \subsection{Herzscher Dipol}
    Elektrisches Pendel: Kondensator und Spule sind Energiespeicher.
    Wenn das magnetische Feld abgebaut wird, so wird das elektrische aufgebaut und umgekehrt.
    Idealisiert (ohne Reibung) "pendelt" dieses System unendlich lange
    \centering
    \includegraphics[height = 30mm]{src/images/herzscher_dipol.png}

    Dipolmoment $\vec{p}$ (alternierende Richtung durch Wechselspannung):
    \mathbox{\vec{p} = q \vec{l} = \vec{p_0} cos(\omega t)}

    Fernfeld: Entfernt man sich weit vom Sender (Herzschen Dipol), so verschwindet der phasenunterschied zwischen elektrischem und magnetischem Feld.

    \mathbox{\vec{E} = \vec{E_0} cos(\omega t - k r)}
    \mathbox{\vec{H} = \vec{H_0} cos(\omega t - k r)}
    mit Kreisfrequenz $\omega = 2 \pi \nu \rightarrow T = \frac{1}{\nu}$, Wellenzahl $k = \frac{2 \pi}{\lambda}$
    [$\nu = $ Frequenz, $T = $ Periode, $\lambda = $ Wellenlänge]
    \vfill \null \columnbreak

    \subsection*{4.2 Wellengleichung}
    \mathbox{\frac{\partial^2 E}{\partial t^2} = \frac{1}{\varepsilon_0 \mu_0} \frac{\partial^2 E}{\partial z^2}}
    \mathbox{\frac{\partial^2 H}{\partial t^2} = \frac{1}{\varepsilon_0 \mu_0} \frac{\partial^2 H}{\partial z^2}} 

\end{document}