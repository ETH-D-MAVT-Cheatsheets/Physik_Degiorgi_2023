\subsection*{2.2 Elektrischer Fluss $\Psi$}
    Satz von Gauss:
    \begin{align*}
        d\Psi = \overrightarrow{E} \overrightarrow{dA} \rightarrow \Psi = \int \overrightarrow{E} d \overrightarrow{A}, [\Psi] = V \cdot m \\
        \oint \overrightarrow{E} \overrightarrow{dA} = \frac{1}{\varepsilon_0} \int \rho dV , div(\overrightarrow{E}) = \frac{1}{\varepsilon_0} \rho\\
        \oint \overrightarrow{F} \overrightarrow{dA} = \int (\nabla \cdot F) dV \rightarrow \oint \overrightarrow{E} \overrightarrow{dA} = \frac{1}{\varepsilon_0} \int \rho dV
    \end{align*}
    Durch Umstellen kommt man auf folgende Formeln für das elektrische Feld:
    \begin{align*}
        \text{Kugel: } \frac{1}{4 \pi \varepsilon_0}\frac{Q}{r^2} \overrightarrow{e_r}
    \end{align*}
    Innerhalb eines elektrischen Leiters ist das elektrische Feld null (es hebt sich auf)\\
    Für geladene Platten: (Siehe Serie 4 A3) $E = \frac{\rho}{2 \varepsilon_0}$