\subsection*{2.3 Plattenkondensator}
    \mathbox{C_0 = \frac{Q}{U} = \varepsilon_0 \frac{A}{l}, [C] = F = \text{Farad}}
    Mit Dielektrikum gefüllter Plattenkondensator:
    \mathbox{C_m = \varepsilon_m C_0 \xrightarrow{Q = const} U_m = \frac{U_0}{\varepsilon_m}, E_m = \frac{E_0}{\varepsilon_mE_0}{\varepsilon_m}}
    Kugel:
    \mathbox{C = 4 \pi \varepsilon_0 \left( \frac{r_1 r_2}{r_2 - r_1} \right)}
    Kirchhoffsche Regeln in Kondensatoren:
    Knotenregel:
    \mathbox{\sum_k Q_k = 0}
    Maschenregel:
    \mathbox{\sum_i = U_i = \sum_k \frac{Q_k}{C_k}}
    Serieschaltung:
    \mathbox{\frac{1}{C} = \frac{1}{C_1} + \frac{1}{C_2}}
    Parallelschaltung:
    \mathbox{C = C_1 + C_2}

    Ladestrom des Kondensators: $I = I_0 e^{-\frac{t}{RC}}$, somit erreicht der Kondensator niemals seine maximale Kapazität