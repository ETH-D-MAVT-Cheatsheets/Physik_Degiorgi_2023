\subsection{1.1 Ladung Q [C]}

\begin{itemize}
    \item Elementarladung: $q_{Elektron} = e = - 1.602 \cdot 10^{-19}C$
\end{itemize}

Coulomb-Kraft: 

\begin{minipage}{0.53\linewidth}
    \begin{footnotesize}
        \begin{center}
            \mathbox{
                \vec{F_C}=\frac{1}{4\pi\varepsilon_0}\cdot \frac{q_1 \cdot q_2}
                {r^2}\cdot \vec{e_r}            }
        \end{center}
    \end{footnotesize}
\end{minipage}
\begin{minipage}{0.46\linewidth}
    \begin{scriptsize}
        \begin{center}
            \begin{align*}
                \varepsilon_0 &= 8,854\cdot10^{-12}
                \\q_{1/2} &= \text{Punktladungen}
                \\r &= \text{Abstand zw. Punktladungen}
                \\\vec{e_r} &= \text{Einheitsvektor}
            \end{align*}
        \end{center}
    \end{scriptsize}
\end{minipage}
\vspace{1mm}



\begin{itemize}
    \item Ladungen mit gleichem Vorzeichen stossen sich ab.
    \\$F_C<0 \rightarrow \text{abstossend}$,
    $F_C>0 \rightarrow \text{anziehend}$
    \item Ladungen leitender Körper stets an Oberfläche.\\
    $\rightarrow$ Inneres: Ladungs und Feldfrei
\end{itemize}

\subsubsection*{Ladungsdichte}

\begin{tabular}{c c c}
    Liniendichte $\lambda$ & Oberflächendichte $\sigma$ & Volumendichte $\rho$ \\
    $\lambda = \frac{Q}{l} \left[\frac{C}{m}\right]$ & $\sigma = \frac{Q}{A} \left[\frac{C}{m^2}\right]$ & $\rho = \frac{Q}{V} \left[\frac{C}{m^3}\right]$
\end{tabular}

