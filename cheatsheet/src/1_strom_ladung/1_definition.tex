\subsection*{1.1 Definition Strom}

Ladungen mit gleichem Vorzeichen stossen sich ab.\\
zwei unendlich lange parallele Drähte im Abstand 1 m voneinander, die von einem Strom von 1 A gleichsinnig durchflossen werden, ziehen sich mit einer Kraft von $2x10^{-7} N$ pro Meter Leiterlänge an.\\
Elementarladung: $e = 1.602 \cdot 10^{-19}$

\mathbox{I = \frac{dQ}{dt}}
\mathbox{Q = \int\limits_{\Delta t} I dt}

\mathbox{R = \frac{U}{I}, I \sim U}

\begin{tabular}{c c}
    Ohmsche Leiter & nicht-ohmsche Leiter \\
    $I = \frac{U}{R}$ & $R_{\text{diff}} = \frac{dU}{dI}$\\
    plot 1 & plot 2
\end{tabular}

