\subsection{Nutzliche Formeln}
    \begin{empheq}{align*}
        &\textbf{Kräfte}\\
        F                           &\quad \text{Kraft Allgemein}           & \scriptstyle = m \cdot a\\
        F_g                         &\quad \text{Gewichtskraft}             & \scriptstyle = m \cdot g\\
        F_\text{Fed}                &\quad \text{Federkraft}                & \scriptstyle = R \cdot s\\
        F_Z                         &\quad \text{Zentripetalkraft}          & \scriptstyle = m \frac{v^2}{r} = m \omega^2 r \\
        &\textbf{Energie}\\
        E                           &\quad \text{Energie Allgemein}         & \scriptstyle = \vec{F} \cdot \vec{s}\\
        E_\text{pot}                &\quad \text{Potentielle Energie}       & \scriptstyle = m \cdot g \cdot h\\
        E_\text{kin}                &\quad \text{Kinetische Energie}        & \scriptstyle = \frac{1}{2} m \cdot v^2\\
        W = \Delta E                &\quad \text{Zusammenhang Arbeit Energie}\\
        \\
        |\vec{a} \times \vec{b}|    &\quad \text{Kreuzprodukt}              & \scriptstyle = |\vec{a}| \cdot |\vec{b}| \cdot \sin(\alpha)\\
        \ddot{\vec{s}} = \dot{\vec{v}} = \vec{a} &\Rightarrow \vec{s} = \vec{v} \cdot t = \frac{\vec{a}}{2} t^2
    \end{empheq}

    \subsubsection{Rechengesetze für Exponenten \& Logarithmen}
        \begin{minipage}{0.29\linewidth}
            \begin{empheq}{align*}
                B^a \cdot B^b =& \; B^{a + b}\\
                \frac{B^a}{B^b} =& \; B^{a - b}\\
                (B^a)^b =& \; B^{a \cdot b}
            \end{empheq}
        \end{minipage}
        \begin{minipage}{0.69\linewidth}
            \begin{empheq}{align*}
                \log_B (a \cdot b ) =& \; \log_B (a) + \log_B (b)\\
                \log_B \left( \frac{a}{b} \right) =& \; \log_B (a) - \log_B (b)\\
                \log_B (a^r) =& \; r \cdot \log_B (a)
            \end{empheq}
        \end{minipage}
        \begin{align*}
            \textrm{Basiswechsel: } \log_a (x) =& \; \frac{\log_b(x)}{\log_b(a)}
        \end{align*}
