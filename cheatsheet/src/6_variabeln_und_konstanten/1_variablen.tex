\subsection{Variablen}
    \begin{empheq}{align*} % Dipolmoment???
        \vec{B}                              &\quad \text{Magnetische Induktion}             & \scriptstyle T = \frac{W b}{m^2} = \frac{V \cdot s}{m^2} = \frac{kg}{A \cdot s^2} \\
        C                                               &\quad \text{Kapazität}                         & \scriptstyle F = \frac{C}{V} = \frac{A \cdot s}{V} = \frac{A^2 \cdot s^4}{kg \cdot m^2} \\
        D                                               &\quad \text{elek. Flussdichte /}               & \scriptstyle \frac{A \cdot s}{m^2}\\
                                                        &\quad \text{Verschiebungsdichte}               & \\
        \vec{E}                              &\quad \text{e. Feld}                           & \scriptstyle \frac{N}{C} = \frac{V}{m} = \frac{kg \cdot m}{s^3 \cdot A} \\
        E                                               &\quad \text{Energie} \quad 1eV \cdot e = 1J    & \scriptstyle J = Nm = CV = Ws \frac{kg \cdot m^2}{s^2} \\
        \scriptstyle f = \nu = \frac{1}{T}              &\quad \text{Frequenz}                          & \scriptstyle \frac{1}{s} = Hz \\
        \vec{F}                              &\quad \text{Kraft}                             & \scriptstyle N = \frac{V \cdot C}{m} = \frac{kg \cdot m}{s^2} \\
        \vec{H}                              &\quad \text{magn. Feldstärke}                  & \scriptstyle \frac{A}{m} \\
        \vec{I}                              &\quad \text{el. Strom}                         & \scriptstyle A = \frac{C}{s} \\
        \scriptstyle \vec{j} = \frac{I}{A}   &\quad \text{Stromdichte}                       & \scriptstyle \frac{C}{s \cdot m^2} \\
        k                                               &\quad \text{Federkonstante}                    & \scriptstyle \frac{N}{m} \\
        L                                               &\quad \text{Induktivität}                      & \scriptstyle H = \frac{T \cdot m^2}{A} = \frac{V \cdot s}{A} \\
                                                        &                                               & \scriptstyle = \frac{kg \cdot m^2}{A^2 \cdot s^2} \\
        P                                               &\quad \text{Leistung}                          & \scriptstyle W = V \cdot A = \frac{J}{s} \\
        Q                                               &\quad \text{Ladung}                            & \scriptstyle C = A \cdot s \\
        R                                               &\quad \text{el. Widerstand}                    & \scriptstyle \Omega = \frac{V}{A} \\
        S                                               &\quad \text{Siemens}                           & \scriptstyle S = \frac{1}{\Omega} = \frac{A}{V} \\
        T                                               &\quad \text{Periodendauer /}                   & \scriptstyle s \\
                                                        &\quad \text{Schwingungsdauer}                  & \\
        U                                               &\quad \text{Potentialdiff. / Spannung}         & \scriptstyle V = \frac{W}{A} = \frac{J}{C} \\
                                                        &                                               & \scriptstyle = \frac{Nm}{As} = \frac{kg \cdot m^2}{A \cdot s^3} \\
        \vec{v}                              &\quad \text{Geschwindigkeit}                   & \scriptstyle \frac{m}{s} \\
        W                                               &\quad \text{Arbeit}                            & \scriptstyle J = N \cdot m \\
                                                        &                                               & \scriptstyle = \frac{kg \cdot m^2}{s^2} = C \cdot V \\
        Z                                               &\quad \text{Impedanz}                          & \scriptstyle \Omega = \frac{V}{A} = \frac{kg \cdot m^2}{A^2 \cdot s^3} \\
        \varepsilon                                     &\quad \text{Dielektrizitätskonst. Mat.}        & \scriptstyle \frac{C}{V \cdot m} = \frac{A \cdot s}{V \cdot m} \\
        \Psi_E                                          &\quad \text{elek. Fluss}                       & \scriptstyle V \cdot m = \frac{N \cdot m^2}{C} \\
        \Phi_M                                          &\quad \text{magn. Fluss}                       & \scriptstyle Wb = T \cdot m^2 \\
        \Phi                                            &\quad \text{elek. Potential}                   & \scriptstyle [-] \\
        \lambda = \frac{c}{f}                           &\quad \text{Wellenlänge}                       & \scriptstyle m \\
        \mu                                             &\quad \text{magn. Feldk. /}                    & \scriptstyle \frac{V \cdot s}{A \cdot m} \\
                                                        &\quad \text{Permeabilität}                     & \\
        \rho                                            &\quad \text{spez. Widerstand}                  & \scriptstyle \Omega \cdot m \\
        \omega = 2 \pi f                                &\quad \text{Kreisfrequenz}                     & \scriptstyle s^{-1} = Hz \\
        %&\quad \text{} & \scriptstyle  \\
    \end{empheq}