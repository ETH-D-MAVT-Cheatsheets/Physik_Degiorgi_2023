\subsection{5.1 LR-Kreise}
    LR-Kreise bestehen aus einer Gleichstromquelle, einem Schalter, einer Spule und einem Widerstand in Reihe geschalten.\\
    Maschenregel:
    \mathbox{U_0 = U_R(t) + U_L(t)}

    mit $U_R(t) = R I(t)$ und $U_L(t) = L \frac{dI(t)}{dt}$ und geteilt durch $L$
    DGL der Stromstärke und ihre Lösung:\\
    Einschalten:
    \mathbox{\dot{I}(t) + \frac{R}{L} I(t) = \frac{U_0}{L}, I(t) = \frac{U_0}{R} \left(1 - e^{-\frac{R}{L}(t - t_0)}\right)}
    Ausschalten ($U_0 = 0$):
    \mathbox{I(t) = \frac{U_0}{R} \left(e^{-\frac{R}{L}(t - t_0)}\right)}
    Zeitkonstante $\tau$:
    \mathbox{\tau = \frac{L}{R}}