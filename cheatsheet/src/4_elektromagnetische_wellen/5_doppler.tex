\subsection{4.5 Doppler-Effekt}
    \mathbox{\lambda_0 = \frac{c}{f_0}}
    
    Wenn Empfänger sich auf Sender zubewegt:
    \mathbox{f' = f_0 (1 + \frac{v}{c})}

    Wenn Empfänger sich von Sender wegbewegt:
    \mathbox{f' = f_0 (1 - \frac{v}{c})}

    Wenn Sender sich auf Empfänger zubewegt:
    \mathbox{f' = \frac{1}{1 - \frac{v}{c}}}

    Wenn Sender sich von Empfänger wegbewegt:
    \mathbox{f' = \frac{1}{1 + \frac{v}{c}}}

    Für $v << c$ gilt, dass es nicht drauf ankommt ob Sender oder Empfänger sich in Ruhe befindet.

    Für Licht im Vakuum:
    $\beta = \frac{v}{c}$

    Quelle und Empfänger entfernen sich (Redshift):
    \mathbox{f' = f_0 \sqrt{\frac{1 - \beta}{1 + \beta}}}

    Quelle und Empfänger nähern sich (Blueshift):
    \mathbox{f' = f_0 \sqrt{\frac{1 + \beta}{1 - \beta}}}