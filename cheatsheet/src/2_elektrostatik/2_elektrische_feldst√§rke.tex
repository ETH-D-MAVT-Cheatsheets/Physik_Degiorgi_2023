\subsubsection*{Elektrische Feldstärke E $\left[\frac{V}{m}\right]$}
Inhomogen (Punktladung):

\vspace{-1mm}
\begin{minipage}{0.53\linewidth}
    \begin{footnotesize}
        \begin{center}
            \mathbox{
                \overrightarrow{E} = \frac{\overrightarrow{F}}{q_2} = \frac{1}{4\pi\varepsilon_0} \cdot \frac{q_1}{| \overrightarrow{r}|^2} \cdot\overrightarrow{e}
            }
        \end{center}
    \end{footnotesize}
\end{minipage}
\vspace{1mm}
\begin{minipage}{0.46\linewidth}
    \begin{scriptsize}
        \begin{center}
            \begin{align*}
                \varepsilon_0 = &\enspace 8,854\cdot10^{-12}
                \\\overrightarrow{r} = &\text{\enspace Abstand zw. Punktladungen}
                \\&\text{\enspace(- nach +)}
                \\\overrightarrow{e_r} = &\text{\enspace Einheitsvektor}
            \end{align*}
        \end{center}
    \end{scriptsize}
\end{minipage}
\vspace{2mm}
$\longrightarrow$ Feld mehrer Punktladungen Summierbar (Vektorsumme)\\
Homogene (Plattenkondensator):

\begin{minipage}{0.53\linewidth}
    \begin{footnotesize}
        \begin{center}
            \mathbox{
                \overrightarrow{E} = \frac{U}{l} \overrightarrow{e}
            }
        \end{center}
    \end{footnotesize}
\end{minipage}
\begin{minipage}{0.46\linewidth}
    \begin{scriptsize}
        \begin{center}
            \begin{align*}
                l &= \text{Abstand der Platten}
            \end{align*}
        \end{center}
    \end{scriptsize}
\end{minipage}
\vspace{1mm}