\subsubsection*{Elektrisches Potential/ Spannung im E-Feld}
Inhomogen: \mathbox{U = \int\overrightarrow{E}\overrightarrow{ds}} 
Beispiel Punktladung:

\vspace{-1mm}
\begin{minipage}{0.53\linewidth}
    \begin{footnotesize}
        \begin{center}
            \mathbox{
                U = \int_{1}^{2}\overrightarrow{E}(r)\:\overrightarrow{dr}            }
        \end{center}
    \end{footnotesize}
\end{minipage}
\vspace{1mm}
\begin{minipage}{0.46\linewidth}
    \begin{scriptsize}
        \begin{center}
            \begin{align*}
                \overrightarrow{E}\text{(r), } &\text{E-Feld wird gegen aussen}
                \\ &\text{schwächer}
            \end{align*}
            \begin{align*}
                \varphi &= \text{Potential} [V]
            \end{align*}
        \end{center}
    \end{scriptsize}
\end{minipage}

\vspace{-1mm}
    \begin{minipage}{0.41\linewidth}
        \begin{footnotesize}
            \begin{center}
                \vspace{2mm}
                \includegraphics[width = 27mm]{src/images/inhom_potentialfeld.png}
            \end{center}
        \end{footnotesize}
    \end{minipage}
    \begin{minipage}{0.58\linewidth}
        \begin{scriptsize}
            \begin{center}
                \mathbox{
                    U = \varphi_0 - \varphi_1 = \int_{1}^{2}E\cdot\cos (\alpha) ds
                }
                Auf \colorbox{Cyan}{Kreis $\varphi_{0/1}$} stets selbes Potential/Spannung
            \end{center}
        \end{scriptsize}
    \end{minipage}
    \vspace{1mm}
\newpage
Homogen:

\vspace{-1mm}
    \begin{minipage}{0.41\linewidth}
        \begin{footnotesize}
            \begin{center}
                \vspace{2mm}
                \includegraphics[width = 27mm]{src/images/hom_potentialfeld.png}
            \end{center}
        \end{footnotesize}
    \end{minipage}
    \begin{minipage}{0.58\linewidth}
        \begin{scriptsize}
            \begin{center}
                \mathbox{
                    U = E\cdot d
                }
                Auf \colorbox{Cyan}{Linie} stets selbes Potential/Spannung
            \end{center}
        \end{scriptsize}
    \end{minipage}
    \vspace{1mm}