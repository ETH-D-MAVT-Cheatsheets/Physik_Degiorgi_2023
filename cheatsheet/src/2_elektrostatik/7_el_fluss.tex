\subsection*{2.2 Elektrischer Fluss $\Psi$}
    \begin{minipage}{0.49\linewidth}
        \mathbox{
            \Psi = \int \overrightarrow{E} d \overrightarrow{A}
        }
    \end{minipage}
    \begin{minipage}{0.49\linewidth}
        \begin{scriptsize}
            $E$ = Elektrisches Feld\\
            $A$ = Fläche, durch die das Feld hindurchfliesst
        \end{scriptsize}
    \end{minipage}
    \begin{minipage}{0.49\linewidth}
        \begin{empheq}[box = \fbox]{align*}
            \oint \overrightarrow{E} \overrightarrow{dA} = \frac{1}{\varepsilon_0} \int \rho dV\\
            div(\overrightarrow{E}) = \frac{1}{\varepsilon_0} \rho
        \end{empheq}
    \end{minipage}
    \begin{minipage}{0.49\linewidth}
        \begin{scriptsize}
            Mittels dem Satz von Gauss erhält man die erste Maxwell Gleichung\\
            $\rho$ = Ladungsdichte\\
        \end{scriptsize}
    \end{minipage}

    Für geladene Platten: (Siehe Serie 4 A3) $E = \frac{\rho}{2 \varepsilon_0}$