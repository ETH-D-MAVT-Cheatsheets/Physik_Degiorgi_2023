\subsection{Elektrischer Fluss $\Psi$}
    \begin{minipage}{0.49\linewidth}
        \mathbox{
            \Psi = \int \vec{E} d \vec{A}
        }
    \end{minipage}
    \begin{minipage}{0.49\linewidth}
        \begin{scriptsize}
            $E$ = Elektrisches Feld\\
            $A$ = Fläche, durch die das Feld hindurchfliesst
        \end{scriptsize}
    \end{minipage}

    \begin{minipage}{0.54\linewidth}
        \begin{empheq}[box = \fbox]{align*}
            \oint \vec{E} \vec{dA} = \frac{1}{\varepsilon_0} \int \rho dV = \frac{Q}{\varepsilon_0}\\
            div(\vec{E}) = \frac{1}{\varepsilon_0} \rho
        \end{empheq}
    \end{minipage}
    \begin{minipage}{0.44\linewidth}
        \begin{scriptsize}
            Mittels dem Satz von Gauss erhält man die erste Maxwell Gleichung\\
            $\rho$ = Ladungsdichte\\
            $Q$ = Gesamtladung innerhalb der Fläche
        \end{scriptsize}
    \end{minipage}

    