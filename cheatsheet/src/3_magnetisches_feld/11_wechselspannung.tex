\subsection*{3.11 Wechselspannung}
    Dynamo (Wechselspannung durch sich drehende Schleife in homogenem Magnetfeld):
    \mathbox{U_i = \omega B_0 A_0 sin(\omega t) = U_0 sin(\omega t)}
    \mathbox{I(t) = I_0 (cos(\omega t) + i sin(\omega t)) = I_0 e^{i \omega t}}

    Widerstand: Phasenverschiebung $U(t) \sim I(t)$
    \mathbox{U_R(t) = R I(t) = Z_R I(t) e^{i \omega t} \text{ mit } Z_R = R}

    Spule: Phasenverschiebung $U(t) \sim I(t + \frac{\pi}{2})$
    \mathbox{U_L = L \frac{dI}{dt} = Z_L I(t) e^{i \omega t} \text{ mit } Z_L = i \omega L}

    Kondensator: Phasenverschiebung $U(t) \sim I(t - \frac{\pi}{2})$
    \mathbox{U_C = \frac{1}{C} \int I(t) dt = Z_C I(t) e^{i \omega t} \text{ mit } Z_C = \frac{1}{i \omega C}}

    Impedanz:
    \mathbox{Z = Re + i Im = |Z| e^{i \phi}}
    \mathbox{U(t) = Z I(t)}
    \mathbox{Z_{\text{tot}} = \sum_i Z_i, \frac{1}{Z_{\text{tot}}} = \sum_i \frac{}{Z_i}}

    Leistung in komplexer Schreibweise ($\rho = $ Phasenverschiebung):
    \mathbox{\frac{1}{2} I_0 U_0 cos(\rho)}
