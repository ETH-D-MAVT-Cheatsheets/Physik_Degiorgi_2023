\subsection{Magnetische Flussdichte B \hfill $[T]$}
    \begin{minipage}{0.49\linewidth}
        \begin{empheq}[box = \fbox]{align*}
            B = \frac{\Phi}{A} = \mu H
        \end{empheq}  
    \end{minipage}
    \begin{minipage}{0.49\linewidth}
        \begin{scriptsize}
            \begin{empheq}{align*}
                \mu &= \mu_0 \cdot \mu_r = \text{Mag. Permeabilität}\\
                A &= \text{Durchflossene Fläche}\\
                H &= \text{Mag. Feldstärke}\\
                \Phi &= \text{Mag. Fluss}
            \end{empheq}
        \end{scriptsize}
    \end{minipage}
    \vspace{2mm}

    \begin{minipage}{0.49\linewidth}
        \centering \underline{\textbf{Langer Leiter}}\\
        \begin{empheq}[box = \fbox]{align*}
            B = \frac{\mu_0 I}{2 \pi r}
        \end{empheq}  
    \end{minipage}
    \begin{minipage}{0.49\linewidth}
        \centering \underline{\textbf{Spule}}\\
        \begin{empheq}[box = \fbox]{align*}
            B = \mu_0 \frac{n}{l}
        \end{empheq}  
    \end{minipage}