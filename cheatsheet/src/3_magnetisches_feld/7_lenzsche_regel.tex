\subsection{Lenzsche Regel}
    \begin{itemize}
        \item System reagiert der zeitlichen Änderung des magnetischen Feldes entgegen / wehrt sich gegen Änderung des magnetischen Feldes
        \item Bedingung: Strom muss in System fliessen können (Bsp metallischer Ring ohne Lücken)
        \item Richtung des induzierten Feldes entgegengesetzt der Richtung der Änderung des äusseren Feldes.
    \end{itemize}
%
    \centering Aus Induktionsgesetz folgt:\\
    \begin{minipage}{0.49\linewidth}
        \begin{empheq}[box = \fbox]{align*}
            \oint \vec{E} \vec{ds} &= -\frac{d}{dt} \int \vec{B} \vec{dA}\\
            rot(\vec{E}) &= -\dot{\vec{B}}
        \end{empheq}
    \end{minipage}
    \begin{minipage}{0.49\linewidth}
        \begin{scriptsize}
            \begin{empheq}{align*}
                \vec{E} &= \text{Elektrische Feldstärke}\\
                \vec{B} &= \text{Magnetische Flussdichte}\\
            \end{empheq}
        \end{scriptsize}
    \end{minipage}