\subsection{elektromagnetische Induktion}
    \centering \underline{\textbf{Spannungsstoss $S_U$}}\\
    Änderung des magnetischen Feldes erzeugt Strom in Spule
    \begin{minipage}{0.44\linewidth}
        \begin{empheq}[box = \fbox]{align*}
            S_U &= \int U_i dt\\
            &= \mu n_i \Delta H A_i
        \end{empheq}  
    \end{minipage}
    \begin{minipage}{0.54\linewidth}
        \begin{scriptsize}
            \begin{empheq}{align*}
                \mu &= \mu_0 \cdot \mu_r = \text{Mag. Permeabilität}\\
                U_i &= \text{Spannung in der Spule i}\\
                n_i &= \text{Windungen der Spule i}\\
                H &= \text{Mag. Feldstärke}\\
                A_i &= \text{Querschnittsfläche der Spule i}\\
            \end{empheq}
        \end{scriptsize}
    \end{minipage}
    
    \centering \underline{\textbf{Induktionsgesetz}}\\ \label{Induktionsgesetz}
    \begin{minipage}{0.58\linewidth}
        \begin{empheq}[box = \fbox]{align*}
            U_i &= -n_i \frac{d\Phi}{dt}\\
            &= \underbrace{-n_i \frac{d}{dt} \iint \vec{B}\vec{dA}}_{\text{$\vec{B}$ hom. auf $A$ und $\vec{B} \| \vec{dA}$,}}\\
            &= -n_i B A
        \end{empheq}
    \end{minipage}
    \begin{minipage}{0.40\linewidth}
        \begin{scriptsize}
            Minus wegen Lenzscher Regel
            \begin{empheq}{align*}
                U_i &= \text{Induzierte Spannung in Spule}\\
                n_i &= \text{Windungen der Spule i}\\
                \Phi &= \text{Mag. Fluss}\\
                B &= \text{mag. Flussdichte}\\
                A &= \text{Querschnittsfläche Spule,}\\ 
                &\text{normal zu Fläche}
            \end{empheq}
        \end{scriptsize}
    \end{minipage}

    \centering \underline{\textbf{Induktionsspule}}\\
    \begin{minipage}{0.44\linewidth}
        \begin{empheq}[box = \fbox]{align*}
            \int U_i dt &= \alpha \int \vec{H} \cdot \vec{ds}\\
            \alpha &= \mu \frac{n_i}{l} A_i\\
            B &= \frac{L I}{n A}
        \end{empheq}
    \end{minipage}
    \begin{minipage}{0.54\linewidth}
        \begin{scriptsize}
            \begin{empheq}{align*}
                U_i &= \text{Spannung in der Spule i}\\
                \vec{H} &= \\
                \mu &= \mu_0 \cdot \mu_r = \text{Mag. Permeabilität}\\
                n_i &= \text{Windungen der Spule i}\\
                l &= \text{Länge der Spule}\\
                B &= \text{mag. Flussdichte}\\
                L &= \text{Induktivität}
                A &= \text{Querschnittsfläche Spule, normal zu Fläche}
            \end{empheq}
        \end{scriptsize}
    \end{minipage}
