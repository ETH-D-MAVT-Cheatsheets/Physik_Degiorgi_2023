\subsection*{3.2 elektromagnetische Induktion}
    Strom in Spule wird durch Änderung des magnetischen Feldes erzeugt.
    Spannungsstoss
    \mathbox{S_U = \int U_i dt}
    $ S_u \sim \Delta H, S_u \sim n_i, S_u \sim A_i$
    
    % Zusammenhang für zwei Spulen verständlicher machen

    \mathbox{\int U_i = \mu_0 n_i \Delta H A_i}
    $\mu_0$ magnetische Permabilität des Vakuums (siehe Basics)\\
    Wenn eine kleinere Spule $S_2$ in einer Grösseren $S_1$ liegt: 
    \mathbox{U_{\text{ind}} = - N_2 A_2 \mu_0 \frac{\Delta I}{\Delta t} \frac{N_1}{l_1}}

    magnetischer Fluss $\Phi$, magnetische Flussdichte $B$:
    \mathbox{\Phi = \mu_0 A H \rightarrow B = \frac{\Phi}{A} = \mu_0 H}
    \mathbox{\overrightarrow{B} = \mu_0 \overrightarrow{H}}
    \mathbox{\Phi = \int \overrightarrow{B} \overrightarrow{dA}}
    \mathbox{U_i = n \frac{d\Phi}{dt}}

    Induktionsspule: \mathbox{\int U_i dt = \alpha \int \overrightarrow{H} \cdot \overrightarrow{ds}, \alpha = \mu_0 \frac{N}{L} A}
    Geschlossener Weg: \mathbox{\oint \overrightarrow{H} \cdot \overrightarrow{ds} = \int \overrightarrow{j} \overrightarrow{dA} = I \cdot n} Stromstärke mal Anzahl Windungen