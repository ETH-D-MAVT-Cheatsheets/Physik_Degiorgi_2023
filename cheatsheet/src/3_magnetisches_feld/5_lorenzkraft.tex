\subsection*{3.5 Lorenzkraft}
    $l$ Länge des Stromdurchflossenen Leiters im Magnetfeld, $\overrightarrow{B}$ Magnetfeld
    \mathbox{\overrightarrow{F_L} = I (\overrightarrow{l} \times \overrightarrow{B})}
    
    mit $V = A \cdot l$ und $j = \frac{I}{A}$:
    \mathbox{\frac{\Delta F}{\Delta A} = \overrightarrow{j} \times \overrightarrow{B} \rightarrow \overrightarrow{F} = \int \overrightarrow{j} \times \overrightarrow{B} dV}

    mit $I = \rho A v$ und somit $\overrightarrow{j} = \rho \overrightarrow{v}$ (v Geschwindigkeit der Ladungen):
    \mathbox{\overrightarrow{F_L} = \int \rho (\overrightarrow{v} \times \overrightarrow{B}) dV = q (\overrightarrow{v} \times \overrightarrow{B})}
    ! Vorzeichen q!

    \subsubsection*{Beispiel Elektromotor}
        \textbf{Bild Einfügen}
        \mathbox{\overrightarrow{M} = \overrightarrow{r} \times \overrightarrow{F} \rightarrow M = I (\overrightarrow{A} \times \overrightarrow{B})}
        Volle Drehung wird nur erreicht mit Umkehrung der Polarisierung des Stroms bei jeder halben Umdrehung. Hierfür wird ein Kommutator verwendet
    
    \subsubsection*{Beispiel parallele stromdurchflossene Drähte}
        \textbf{Bild Einfügen}
        \mathbox{F_1 = F_2 = \frac{\mu_0}{2 \pi} l \frac{I_1 I_2}{r}}
        
