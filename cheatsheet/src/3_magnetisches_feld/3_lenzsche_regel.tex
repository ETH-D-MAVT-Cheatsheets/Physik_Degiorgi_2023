\subsection{3.3 Lenzsche Regel}
    \begin{itemize}
        \item System reagiert der zeitlichen Änderung des magnetischen Feldes entgegen / wehrt sich gegen Änderung des magnetischen Feldes
        \item Bedingung: Strom muss in System fliessen können (Bsp metallischer Ring ohne Lücken)
        \item Richtung des induzierten Feldes entgegengesetzt der Richtung der Änderung des äusseren Feldes.
    \end{itemize}
%
    \centering Aus Induktionsgesetz folgt:\\
    \begin{minipage}{0.49\linewidth}
        \begin{empheq}[box = \fbox]{align*}
            \oint \overrightarrow{E} \overrightarrow{ds} &= -\frac{d}{dt} \int \overrightarrow{B} \overrightarrow{dA}\\
            rot(\overrightarrow{E}) &= -\dot{\overrightarrow{B}}
        \end{empheq}
    \end{minipage}
    \begin{minipage}{0.49\linewidth}
        \begin{scriptsize}
            \begin{empheq}{align*}
                \overrightarrow{E} &= \text{Elektrische Feldstärke}\\
                \overrightarrow{B} &= \text{Magnetische Flussdichte}\\
            \end{empheq}
        \end{scriptsize}
    \end{minipage}